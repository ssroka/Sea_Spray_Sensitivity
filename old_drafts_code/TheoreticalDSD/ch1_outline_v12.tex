 \documentclass[10pt,a4paper]{article}
\usepackage[utf8]{inputenc}
\usepackage{amsmath}
\usepackage{amsfonts}
\usepackage{amssymb}
\usepackage{multicol}

% bibliography
\usepackage[authoryear]{natbib}

% figures
\usepackage{graphicx}
\usepackage{subcaption}
\usepackage{wrapfig}

%text format
\usepackage{color}


% document propoerties format
\usepackage[margin=0.75in]{geometry}
\usepackage{fancyhdr}
\usepackage{indentfirst}
\usepackage{multicol}

\begin{document}

 \renewcommand{\theenumi}{\Roman{enumi}}
 \renewcommand{\theenumii}{\arabic{enumii}}
 \renewcommand{\theenumiii}{\alpha{enumiii}}
 
\section{Abstract}
\large
\begin{enumerate}
\item New approximation for radius and temperature evolution
\item New approximation with the latest published SGF's show that the spray can support much more of the enthalpy flux than previously thought
\item New approximation formulas allow for a scaling argument for 'most efficient drop' - which has no bearing on the drops that are actually created.
\item New approximation is used to investigate the range of possible SGF's.
\end{enumerate}

 \section{Background}
 
It was shown in \citet{Andreas2001} that re-entrant sea spray accounted for enhanced enthalpy flux that was necessary for models to realize moderate to strong hurricanes. This work builds on those ideas, but rather than allowing the dynamics of a 100$\mu$m drop to be representative of the net flux, new approximation formulas describe the thermodynamics of sea spray based on the environmental conditions and allow for a refinement of the total heat flux. Recently published spray generating functions (SGFs) used in conjuction with this new approximation show that the sea spray can support much more of the enthalpy flux than previously thought; this analysis suggests that up to 100\% of the enthalpy flux may come from sea spray.

\section{New approximation for Temperature Evolution}

As in previous work \citep{Andreas1990,Andreas2001}, the case of a salt-water drop ejected into constant-temperature air is considered. After a short time aloft the drop cools to its salinity-adjusted, wet-bulb temperature typically evaporating less than 1\% of its mass in the process. Mass does leave the surface of the drop during this evaporative cooling characterized by as the competing effects of thermal diffusion and vapor pressure disequilibrium. The result is that the drop achieves a temperature lower than that of the ambient environment. After a comparatively long time, the drop will begin to evaporate a significant fraction of its mass as it absorbs latent heat from the air. As elucidated in \citet{Andreas2001}, since most drops are likely to fall back to the sea before they have absorbed much latent heat, these re-entrant drops enhance the enthalpy flux. Another way to understand this enthalpy enhancement is that when the drops re-enter the sea, they are substantially cooler than they were upon being ejected, thus cooling the sea with the same force with which they heated the air. Figure \ref{fig:singleDrop} of a single drop evolution is created using the framework described in \citet{Andreas1990}, which is largely informed by the microphysics described in \citet{Pruppacher1978}. This framework describes the set of coupled, highly nonlinear equations which govern the drop's thermodynamics during evaporation. This calculation differs from \citet{Andreas1990} in that updated sea water properties from \citet{Nayar2016} are used. The formulas for sensible ($Q_s$) and latent ($Q_L$) heat transferred from the drop to the air are
\begin{align}
q_s &= \rho c_p \left(T_s - T(t)\right)\left(\frac{4}{3}\pi r_0^3\right)\\
Q_s &= q_s \frac{dF}{dr}\label{eq:Qs}\\
q_L &= -\rho L_v \left(1 - \frac{r(t)^3}{r_0^3}\right)\left(\frac{4}{3}\pi r_0^3\right)\\
Q_L &= q_L \frac{dF}{dr}\label{eq:QL}
\end{align}
where $\rho$ is the density of sea water, $c_p$ is the specific heat capacity of sea water, $T_s$ is the ambient temperature of sea water, $T$ is the temperature of the drop, $r_0$ is the initial radius of the drop, $L_v$ is the latent heat of vaporization, and $r$ is the radius of the drop. The energy transfer in Joules is represented by $q_s$ and $q_L$, while the sensible and latent heat fluxes in Watts per square meter per micrometer are $Q_s$ and $Q_L$, respectively. The sea spray generating function $\frac{dF}{dr}$ is the number of drops ejected per square meter, per second, per drop radius. \citet{Andreas2001} explains how a drop which falls back to the sea while the temperature is relatively constant near the wet bulb, and the radius is relatively constant near the initial radius, will have transferred a lot of sensible heat without absorbing much latent heat such that the enthalpy transfer to the air ($q_s+q_L$) is positive.

\begin{figure}[h!]
    \centering
    \begin{subfigure}[t!]{0.75\textwidth}
        \includegraphics[width=\textwidth]{imgs/singleDrop.png}        
    \end{subfigure}
    ~ %add desired spacing between images, e. g. ~, \quad, \qquad, \hfill etc. 
      %(or a blank line to force the subfigure onto a new line)
    \begin{subfigure}[t!]{0.2\textwidth}
        \includegraphics[width=\textwidth]{imgs/singleDropLegend.png}        
    \end{subfigure}
    ~ %add desired spacing between images, e. g. ~, \quad, \qquad, \hfill etc. 
    %(or a blank line to force the subfigure onto a new line)
       \caption{The evolution of a single 100$\mu$m drop where RH = 90\% and $\Delta$T is 3K.  }
       \label{fig:singleDrop}
\end{figure}

The net energy in Joules transferred to the air from the drop is a function of the time the drop spends aloft; the residency time that will transfer the most enthalpy is that which corresponds to the drop having just achieved its wet-bulb temperature. Drops with larger initial sizes will take longer to cool to their wet-bulb temperature, but can affect more enthalpy transfer because of their larger mass, as shown in Figure \ref{fig:qsqL}. 

\begin{figure}[h!]
    \centering
        \includegraphics[width=\textwidth]{imgs/qsqL_drops.png}        
       \caption{The energy transfer from drops of different initial radii for different flight times where RH = 90\% and $\Delta$T is 3K. }
       \label{fig:qsqL}
\end{figure}

As is expected, the evolution of a drop's radius and temperature are influenced by the ambient conditions. Figures \ref{fig:changeRH},\ref{fig:changeDT}, and \ref{fig:changer0} show the effects of the relative humidity ($RH$), air-sea temperature difference ($\Delta T$), and the initial drop radius ($r_0$), respectively. 

\begin{figure}[h!]
    \centering
    \begin{subfigure}[t!]{0.75\textwidth}
        \includegraphics[width=\textwidth]{imgs/changeRH.png}        
    \end{subfigure}
    ~ %add desired spacing between images, e. g. ~, \quad, \qquad, \hfill etc. 
      %(or a blank line to force the subfigure onto a new line)
    \begin{subfigure}[t!]{0.2\textwidth}
        \includegraphics[width=\textwidth]{imgs/changeRH_Legend.png}        
    \end{subfigure}
    ~ %add desired spacing between images, e. g. ~, \quad, \qquad, \hfill etc. 
    %(or a blank line to force the subfigure onto a new line)
       \caption{As the relative humidity increases, the wet bulb temperature increases, and the drop takes longer to evaporate the same amount of its radius. The dashed, vertical line indicates the expected time of flight according to equation \ref{eq:tauf} for $U_{10} = 50$m/s.  \label{fig:changeRH}}
\end{figure}

\begin{figure}[h!]
    \centering
    \begin{subfigure}[t!]{0.75\textwidth}
        \includegraphics[width=\textwidth]{imgs/changeDT.png}        
    \end{subfigure}
    ~ %add desired spacing between images, e. g. ~, \quad, \qquad, \hfill etc. 
      %(or a blank line to force the subfigure onto a new line)
    \begin{subfigure}[t!]{0.23\textwidth}
        \includegraphics[width=\textwidth]{imgs/changeDTLegend.png}        
    \end{subfigure}
    ~ %add desired spacing between images, e. g. ~, \quad, \qquad, \hfill etc. 
    %(or a blank line to force the subfigure onto a new line)
       \caption{As the difference between the sea and air temperature increases, the wet bulb temperature increases, the disparity between the drop's initial temperature and its wet bulb temperature increases. The radius evolution is comparitively unaffected by an increase in $\Delta T$.\label{fig:changeDT}}
\end{figure}

\begin{figure}[h!]
    \centering
    \begin{subfigure}[t!]{0.75\textwidth}
        \includegraphics[width=\textwidth]{imgs/changer0.png}        
    \end{subfigure}
    ~ %add desired spacing between images, e. g. ~, \quad, \qquad, \hfill etc. 
      %(or a blank line to force the subfigure onto a new line)
    \begin{subfigure}[t!]{0.2\textwidth}
        \includegraphics[width=\textwidth]{imgs/changer0Legend_1.png}        
    \end{subfigure}
    ~ %add desired spacing between images, e. g. ~, \quad, \qquad, \hfill etc. 
    %(or a blank line to force the subfigure onto a new line)
       \caption{Drops of different radii in the same ambient conditions will achieve very nearly the same wet-bulb temperature, but larger drops take longer to reach their wet bulb temeprature, longer to evaporate, and have shorter residency times. \label{fig:changer0}}
\end{figure}

The estimated time of flight $\tau_f$ comes from \citet{Andreas1992} 
\begin{align}
\tau_f = \frac{0.015 U_{10}^2}{u_f} \label{eq:tauf}
\end{align}
for 10m wind speed $U_{10}$ and Stokes fall speed $u_f$. \par 
Since the radius and temperature evolve in a predictable way with respect to the environmental parameters, it is possible to construct a fit for these two drop characteristics that is very reliable at least until the drop is predicted to re-enter the sea. The temperature evolution is modeled as a sum of exponentials, where the coefficients depend on both $RH$ and $\Delta T$.

\begin{align}
T(t) &= c_1e^{-t/\tau_1}+c_2e^{-t/\tau_2} \\
c_1 &= \alpha_1 RH + \alpha_2 \Delta T + \alpha_3 \\
c_3 &= \alpha_4 RH + \alpha_5 \Delta T + \alpha_6  \\
\begin{split}
\tau_1 &= (a_7RH+a_8 \Delta T+a_9) e^{a_{10}RH+a_{11}\Delta T+a_{12})r_0}+\\
&(a_{13}RH+a_{14} \Delta T+a_{15}) e^{a_{16}RH+a_{17} \Delta T+a_{18})r_0}
\end{split}\\
\tau_2 &= (\alpha_{19} RH+\alpha_{20} \Delta T+\alpha_{21})/r_0^2+\alpha_{22} 
\end{align}
Since the radius evolution is relatively linear in the temporal region of interest, the radius evolution can be modeled with a linear equation where the coefficients are purely functions of $r_0$.
\begin{align}
 r(t) &= c_3 t + c_4 \\
c_3 &= (\beta_1 RH+\beta_2 \Delta T+\beta_3)\frac{1}{r_0}+(\beta_4 RH^2+\beta_5 RH+\beta_6 \Delta T+\beta_7) \\
c_4 &= \beta_3 r_0+\beta_4 
\end{align}
The constants for these fits can be found in Table \ref{tab:coeffs}.

\begin{table}[!h]
\centering
\hspace*{-0.25in}\begin{tabular}{rlrlrlrlrlrl}
$\alpha_{1}=$ & 0.140 & $\alpha_{2}=$ & -0.926 & $\alpha_{3}=$ & 15.141 & $\alpha_{4}=$ & -0.137 & $\alpha_{5}=$ & 0.917 &  \\
$\alpha_{6}=$ & 13.644  & $\alpha_{7}=$ & -7.94E-07  & $\alpha_{8}=$ & 9.88E-06 & $\alpha_{9}=$ &  5.61E-05& $\alpha_{10}=$ & -6.651E-06 & \\
  $\alpha_{11}=$  & -2.01E-04 & $\alpha_{12}=$ & 3.58E-03 & $\alpha_{13}=$ & 2.43E-05  & $\alpha_{14}=$ & 4.54E-05 & $\alpha_{15}=$ &  -2.45E-03 & \\
$\alpha_{16}=$ & 3.50E-04&   $\alpha_{17}=$  & 2.97E-03& $\alpha_{18}=$ & -4.60E-02 & $\alpha_{19}=$ & 328.15 & $\alpha_{20}=$ & -2079.22 & \\
$\alpha_{21}=$ &  52,318.8 & $\alpha_{22}=$ &0.013&   & & & \\
 $\beta_{1}=$ & 1.55E-06 & $\beta_{2}=$ & 4.13E-07 & $\beta_{3}=$ & -1.523E-08 & $\beta_{4}=$& 8.19E-12 & $\beta_5=$ & -1.523E-09 \\
 $\beta_6=$ &-5.772E-11 & $\beta_7$= & 7.072E-08 & $\beta_8$= & 1.00E-06 & $\beta_9=$ &6.262E-09\\
\end{tabular}
\caption{Fitted constants for $r(t)$ and $T(t)$. \label{tab:coeffs}}
\end{table}

A few sample evolution parameters using these approximations are provided in Figures \ref{fig:sample_100_5_80}, \ref{fig:sample_500_1_95}, and \ref{fig:sample_1000_3_85}.

\begin{figure}[h!]
    \centering
        \includegraphics[width=\textwidth]{imgs/rT_est_100_5_80.png}        
       \caption{ The evolution of a drop with $r_0 = 100\mu$m, $RH = 80$\%, and $\Delta T$ = 5K, the temperature extimate ('x') and radius estimate ('o') are shown on top of the microphysical model profiles} 
       \label{fig:sample_100_5_80}
\end{figure}
\begin{figure}[h!]
    \centering
        \includegraphics[width=\textwidth]{imgs/rT_est_500_1_95.png}        
       \caption{As Figure \ref{fig:sample_100_5_80}, except with $r_0 = 500\mu$m, $RH = 95$\%, and $\Delta T$ = 1K } 
       \label{fig:sample_500_1_95}
\end{figure}

\begin{figure}[h!]
    \centering
        \includegraphics[width=\textwidth]{imgs/rT_est_1000_3_85.png}        
       \caption{As Figure \ref{fig:sample_100_5_80}, except with $r_0 = 1000\mu$m, $RH = 85$\%, and $\Delta T$ = 3K }
        \label{fig:sample_1000_3_85} 
\end{figure}

 
To evaluate the range and efficacy of the approximation, 460 profiles are shown in Figure \ref{fig:error_prof_div}. Each profile is of the estimated $T$ ($r$) divided by the $T$ ($r$) from the full microphysical model. Each profile only extends in time until each drop's $\tau_f$ according to Equation \ref{eq:tauf}, and the agreement is generally within 1\%. 

\begin{figure}[h!]
    \centering
        \includegraphics[width=\textwidth]{imgs/error_prof_div.png}        
       \caption{The ratio of both the estimated temperature to the microphysical model temperature and the ratio of the estimated radius to the extimated model radius vary for the range of RH, $\Delta T$, and $r_0$ examined, but generally remain within a few percent of unity. \label{fig:error_prof_div}}
\end{figure}
 
To confirm the efficacy of the approximation, $Q_s$ and $Q_L$ are evaluated for several different sea spray generating functions using both points from the full microphysical model and the approximation, the results are shown in Figure \ref{fig:many_SGF_spray_flux_est_exact}.

To get the total flux, Equations \ref{eq:Qs} and \ref{eq:QL} are integrated over the drop radii. The values of $T$ and $r$ are evaluated at $\tau_f$. 

\begin{align}
E = \int Q_s(\tau_f(U_{10},r_0)) dr_0 + \int Q_L(\tau_f(U_{10},r_0)) dr_0\\
\end{align}

The recently published SGF in \citet{Troitskaya2018} finds that the primary mechanism for drop creation in extreme wind speeds is bag-break up, and so there are substantially more large drops in their SGF. 

Figure \ref{fig:many_SGF_spray_flux_est_exact} shows the net heat flux for several different published SGF's using both the full microphysical model and the approximation for $T(t)$ and $r(t)$. 

One interesting conclusion from this analysis is that, while some theories separate the enthalpy flux into an 'interfacial flux' and a 'spray flux', this suggests that potentially the entire enthalpy flux can be supported by the spray alone. This is due to the increased proportion of large drops.

\section{Scaling of Enthalpy Flux} 

Even though larger drops can contribute more enthalpy, as shown in Figure \ref{fig:qsqL}, the wind field has to expend energy to accelerate and elevate drops. The wind will impart kinetic energy ($\frac{1}{2}\rho \frac{4}{3}\pi r^3 v^2$) and gravitational potential energy ($\rho \frac{4}{3}\pi r^3 gh$), both of which are proportional to the mass. An expression for which drop contributes the most energy per unit mass is

\begin{align}
\max& \qquad \frac{q_s + q_L}{\rho \frac{4}{3}\pi r_0^3}\\
\max& \qquad \frac{c_p \left(T_s - T\right)\rho \frac{4}{3}\pi r_0^3  -\rho L_v \left(1 - \frac{r^3}{r_0^3}\right)\frac{4}{3}\pi r_0^3}{\frac{4}{3}\pi r_0^3}\\
\max& \qquad c_p \left(T_s - T\right)  - L_v \left(1 - \frac{r^3}{r_0^3}\right)
\end{align}

This final expression already provides some confirmation of the intuitive conclusion that the faster the drop temperature decreases without a significant change in radius, will provide the most enthalpy. The drop also has to be aloft long enough to realize some significant temperature drop, though from Figure \ref{fig:qsqL} it can be seen that the larger drops always contribute more enthalpy per unit time, until the radius shrinks substantially. Figure \ref{fig:qsqL_div_mass} shows the same calculation as Figure \ref{fig:qsqL}, except normalized by the mass. Note that the full microphysical model is used to acquire both Figures \ref{fig:qsqL} and \ref{fig:qsqL_div_mass}.
\begin{figure}[h!]
    \centering
    \begin{subfigure}[t!]{0.65\textwidth}
        \includegraphics[width=\textwidth]{imgs/qsqL_drops_div_mass.png}        
    \end{subfigure}
    ~ %add desired spacing between images, e. g. ~, \quad, \qquad, \hfill etc. 
      %(or a blank line to force the subfigure onto a new line)
    \begin{subfigure}[t!]{0.65\textwidth}
        \includegraphics[width=\textwidth]{imgs/qsqL_drops_div_mass_tof.png}   
        \caption{Unfortunately, even though $\tau_f$ is monotonic with $r_0$, $q_L$ depends on $(1-(r(\tau_f)/r_0)^3)$ and this term is non-monotonic which causes the apparently noisy behavior. }     
    \end{subfigure}
    ~ %add desired spacing between images, e. g. ~, \quad, \qquad, \hfill etc. 
    %(or a blank line to force the subfigure onto a new line)
       \caption{The drop which contributes the most enthalpy per unit mass is a function of the time spent aloft. \label{fig:qsqL_div_mass}}
\end{figure}

 
 \begin{figure}[h!]
    \centering
        \includegraphics[width=\textwidth]{imgs/many_SGF_spray_flux_est_exact.png}        
       \caption{ \label{fig:many_SGF_spray_flux_est_exact}}
\end{figure}
 
 
\newpage

\bibliographystyle{apa}
\bibliography{../References/bibtex/all_of_everything}
 
 
\end{document}