\documentclass[10pt,a4paper]{article}
\usepackage[utf8]{inputenc}
\usepackage{amsmath}
\usepackage{amsfonts}
\usepackage{amssymb}
\usepackage{multicol}

% bibliography
\usepackage[authoryear]{natbib}

% figures
\usepackage{graphicx}
\usepackage{subcaption}
\usepackage{wrapfig}

%text format
\usepackage{color}


% document propoerties format
\usepackage[margin=1in]{geometry}
\usepackage{fancyhdr}
\usepackage{indentfirst}
\usepackage{multicol}

\begin{document}
\noindent Key Points: \\
1) We can use micro-physics of evaporation to describe the net energy transfer.\\
2) Sea spray can account for 100\% of the heat extraction
\bigskip

 \renewcommand{\theenumi}{\Roman{enumi}}
 \renewcommand{\theenumii}{\arabic{enumii}}
 \renewcommand{\theenumiii}{\alpha{enumiii}}
\begin{enumerate}
\item $\frac{\partial K}{\partial t} = \int Y(r)\phi(r) dr\label{eq:Kdt}$\\
for $\frac{\partial K}{\partial t}$ the total power flux [W/m$^2$], $Y(r)$ the heat transfer in Joules of individual drops, and $\phi(r)$ the sea spray generating function in number of drops per m$^2$ per second per radius. 
\item the CBLAST results let us get energy flux as a function of radius
\begin{enumerate}
\item figure of energy flux as a function of radius
\begin{center}
\includegraphics[width = 3in]{imgs/flux_fxn_r.png}
\end{center}
\end{enumerate}
\item the microphysical equations in \citet{Pruppacher1978} tell us what the heat transfer is from an individual drop\\
\begin{enumerate}
\item equations for the evolution of drop temperature and radius\\
\begin{align*}
\text{Temperature Evolution:}&\hspace*{1cm}\frac{\partial T}{\partial t} &=& \frac{3\Big(k_a'(T_a-T)+L_vD_w'(\rho_v-\rho_{v,r})\Big)}{\rho_sc_{ps}r^2}\\
\vspace*{0.25cm}\\
\text{Radius Evolution:}&\hspace*{1cm}\frac{\partial r}{\partial t} &=& \frac{[(f-1)-y]r^{-1}}{\frac{\rho_sRT_a}{D_w'M_we_{\text{sat}}(T_a)}+\frac{\rho_sL_v}{k_a'T_a}\left(\frac{L_vM_w}{RT_a}-1\right)}.
\end{align*}
\item equations of sensible and latent heat transfer from drops to air 
\begin{align*}
Q_s = c_p(T_0-T(t))\rho_s\frac{4}{3}\pi r_0^3\\
Q_L = L_v\left(1-\left(\frac{r(t)}{r_0}\right)^3\right)\rho_s\frac{4}{3}\pi r_0^3
\end{align*}
The characteristic heat transfer from a single drop (with a radius of 100$\mu$m) is about 10$^{-4}$ Joules, and if a hurricane extracts $\approx 100$ W/m$^{2}$, then we can expect that about one drop with a radius of 100$\mu$m to be created in each mm$^2$ per second, which works out to a volume flux of $\approx 4 10^{-6}$ m$^3$/(m$^2$ s). \\
Independently, we find that the volume flux from a number of observational studies of the sea spray generating function ($S_0$) is \[ \int_r S0 \frac{4}{3}\pi r^3 dr \approx 8 10^{-6} m^3/(m^2 s)\] for $S0$ in units of number of drops created per m$^2$ per second. This scale analysis helps reinforce our confidence that the spray evaporation can account for all of the heat flux.
\item figure of sensible and latent heat exchange for different drop radii
\begin{center}
\hspace*{-2.5in}\includegraphics[scale=.5]{imgs/QsQL_drops.png}
\end{center}
\end{enumerate}
\item It has been analytically shown that filament-break up produces a gamma distribution of drop sizes
\begin{enumerate}
\item see if I can include an image from the Miami tank experiment of drops being created off the back of waves
\end{enumerate}
\item We consider all drops to have a residency of 1 second in the air and a volume flux of $\approx 10 \times 10^{-6} m^3/(m^2 s)$. The magnitude of the heat flux is approximately linear with the volume flux (the shape of the contour plot doesn't change, but if I double the volume flux I double the heat flux).
\item After conducting a parameter search in a range of possible gamma distributions, we find a relatively small amount of distributions that will provide the required head transfer
\begin{enumerate}
\item figure of contour plot of energy
\hspace*{-3in}\includegraphics[width = 3.5in]{imgs/E_contour.png}
\includegraphics[width = 3.5in]{imgs/possibleGammas.png}
\end{enumerate}
\item Incorporating drop collisions may change the distributions slightly.
\end{enumerate}




\bibliographystyle{apa}
\bibliography{../References/bibtex/TheoreticalDSD}
\end{document}